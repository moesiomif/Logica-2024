%Cespe-Sefaz-2024
\item
	No argumento seguinte, as proposições $P_1$, $P_2$, $P_3$ e $P_4$ são as premissas, e $Q$ é a
conclusão.
\begin{enumerate}
	\item $P_1$: ''Se há carência de recursos tecnológicos no setor Alfa, então o trabalho dos servidores públicos que atuam nesse setor ficam prejudicados''.
	\item $P_2$: ''Se há carência de recursos tecnológicos no setor Alfa, então os beneficiários dos serviços prestados por esse setor são mal atendidos''.
	\item $P_3$: ''Se o trabalho dos servidores públicos que atuam no setor Alfa fica prejudicado, então os servidores públicos que atuam nesse setor padecem''.
	\item $P_4$: “Se os beneficiários dos serviços prestados pelo setor Alfa são mal atendidos, então os beneficiários dos serviços prestados por esse setor padecem''.
	\item $Q$: ''Se há carência de recursos tecnológicos no setor Alfa, então os servidores públicos que atuam nesse setor padecem e os beneficiários dos serviços prestados por esse setor padecem''.
\end{enumerate}
Considerando esse argumento, julgue o item seguinte. O argumento em questão é válido? (Justifique)
%\begin{itemize}
%	\item $( )$ CERTO 
%	\item $( )$ ERRADO
%\end{itemize}
