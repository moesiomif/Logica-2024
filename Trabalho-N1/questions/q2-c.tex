\item
	Ao comentar a respeito da qualidade dos
serviços de entrega de uma empresa, um aluno fez as seguintes afirmações:
\begin{itemize}
	\item $P_1$: Se for bom e rápido, não será barato.
	\item $P_2$: Se for bom e barato, não será rápido.
	\item $P_3$: Se for rápido e barato, não será bom.
\end{itemize}
Com base nessas informações, julgue.

Um argumento que tenha $P_1$ e $P_2$ como
premissas e $P_3$ como conclusão será um
argumento válido.

\SOL{Sejam:
	\begin{itemize}
		\item $B$: bom;
		\item $Bt$: barato;
		\item $R$: rápido
	\end{itemize}
	podemos traduzir em linguagem simbólica:
	\[
		(B\wedge R) \rightarrow \neg Bt,\ (B\wedge Bt)\rightarrow \neg R \vdash (R\wedge Bt)  \rightarrow \neg B 
	\]
utilizando o Teorema de Dedução:
	\[
		(B\wedge R) \rightarrow \neg Bt,\ (B\wedge Bt)\rightarrow \neg R,\ (R\wedge Bt)  \vdash  \neg B 
	\]
\begin{center}
		\[\begin{tabular}{lll}
			(1)\ &$(B\wedge R)       \rightarrow \neg Bt$ &\\
			(2)\ &$(B\wedge Bt)    \rightarrow \neg R$ &\\
			(3)\ &$(R\wedge Bt)$ &\\
		         \hline
			(4)\ &$ R  $ & $3- SIMP$ \\
			(5)\ &$ Bt  $ & $3- SIMP$ \\
			(6)\ &$ \neg \neg Bt  $ & $5- DN$ \\
			(7)\ &$ \neg (B \wedge R)  $ & $1,6- MT$ \\
			(8)\ &$ \neg B \vee \neg R  $ & $7- DM$ \\
			(9)\ &$ \neg \neg R  $ & $4- DN$ \\
			(10)\ &$ \neg B  $ & $8,9- SD$ \\
		\end{tabular}\]
		\end{center}
A dedução acima confirmar a validade do argumento.



}
