%        File: Trabalho-2.tex
%     Created: Qua Out 18 05:00  2017 B
% Last Change: seg 26 ago 2024 19:59:53
%                 Author: Moésio M. de Sales
%                   moesio@ifce.edu.br
%---------------------------------------------
\documentclass[11pt,a4paper]{article}
\usepackage{ifce}
%---------------------------------------------
\newcommand{\nome}{\bf Mo\'esio M. de Sales\footnote{moesio@ifce.edu.br}}
\newcommand{\titu}{Trabalho - \texorpdfstring{$7$}{} Pontos}
\newcommand{\subtitu}{Trabalho I M\'etodo Dedutivo}
\newcommand{\palachav}{M\'etodo Dedutivo; Redução do número de Conectivo; Forma normal das
proposições; CNC; FND; Princípio da Dualidade}
\newcommand{\disc}{Lógica Matemática}
\newcommand{\curso}{Sistemas de Informação}
\newcommand{\inst}{IFCE}
\newcommand{\instr}{\today}
\newcommand{\cabl}{Esquerdo}
\newcommand{\cabc}{Centro}
\newcommand{\cabr}{Direita}
\title{\titu}
\author{\nome}
\date{\today}
\hypersetup{pdfsubject={\subtitu},pdfkeywords={\palachav}}
%---------------------------------------------
\usepackage{packageifce}
%---------------------------------------------
\begin{document}
{\Large
\begin{center} \titu\\ \disc\\ \curso\\ \nome\end{center}
}
%---------------------------------------------
\hrule
\noindent
{\bf Alun@:}\hrulefill\

%---------------------------------------------
%---------------------------------------------
\section{\sc Justifique suas Respostas; na justificativa sempre represente as proposições e provas de forma simbólica.}
%---------------------------------------------
%---------------------------------------------
\exerc{%
\item
	Indicar a ''{{\sc Regra de Inferência }}'' que justifica a validade dos seguintes argumentos:
		\begin{enumerate}
				\item $p\to q,\ r\to \neg s \vdash (p\to q)\wedge (r\to \neg s)$
				\item $p \to q \vee r \vdash p \to p\wedge  ( q \vee r)$
		\end{enumerate}

\item Prove:
	\[
	(P\vee Q) \wedge (P \vee R), P\to S, Q\to S, P\to T, R\to T \vdash S \wedge T
\]

\item Usar a regra:
	\begin{enumerate}
\item
		do ''{\sc Dilema Construtivo}'' para deduzir a conclusão do argumento dado pelas seguintes ternos de premissas:
\[
				\begin{tabular}{ll}
						$(1)$ & $x=2 \rightarrow x^2= 4$ \\
						$(2)$ & $x=2 \vee y= 3$ \\
						$(3)$ & $y=3 \rightarrow y^2= 9$ 
					   %$(2)$ & $p\wedge q$ & P\\ \hline
				\end{tabular}
\]		

\item
		do ''{\sc Dilema Destrutivo}'' para deduzir a conclusão do argumento dado pelas
seguintes ternos de premissas:
\[
				\begin{tabular}{ll}
						$(1)$ & $y\neq 9 \vee y\neq 18$ \\
						$(2)$ & $x=2 \rightarrow y= 9$ \\
						$(3)$ & $x=8 \rightarrow y= 18$ 
					   %$(2)$ & $p\wedge q$ & P\\ \hline
				\end{tabular}
\]
	\end{enumerate}

\item Determine se o argumento é válido ou inválido:

     Premissas:

\begin{enumerate}
\item Se eu lesse o jornal na cozinha, meus óculos estariam na mesa da cozinha.
\item Não li o jornal na cozinha.
\end{enumerate}
		            Conclusão: meus óculos não estão na mesa da cozinha.

\item Determine se o argumento é válido ou inválido.

Premissas:
\begin{enumerate}
\item Se eu não estudo muito, não vou passar neste curso.
\item Se eu não passar este curso não posso me formar este ano.
\end{enumerate}

Conclusão: se eu não estudo muito, não vou me formar este ano.

\item
		 Um grande banco formará uma comissão que será comandada por três de suas funcionárias, Alice, Beatriz e Carla. Nessa comissão, uma delas será a presidente, outra será a gerente, e outra, a coordenadora. A distribuição desses cargos deverá, necessariamente, considerar as quatro seguintes restrições:
		 \begin{itemize}
				 \item (I) Se Carla for a gerente, então Beatriz terá de ser a presidente;
				 \item (II) Se Alice for a presidente, então Carla terá de ser a gerente;
				 \item (III) Se Beatriz não for a coordenadora, então Alice terá de ser a gerente;
				 \item (IV) Se Carla for a coordenadora, então Beatriz terá de ser a gerente.
		 \end{itemize}

Nessas circunstâncias, os respectivos cargos de Alice, Beatriz e Carla serão
\begin{enumerate}
		\item gerente, coordenadora e presidente 
		\item gerente, presidente e coordenadora
		\item coordenadora, presidente e gerente
		\item presidente, gerente e coordenadora
		\item presidente, coordenadora e gerente
\end{enumerate}

\noindent
\SOL{Sejam:
	\begin{itemize}
		\item $C_x$: Carla é $x$;
		\item $B_x$: Beatriz é $x$;
		\item $A_x$: Alice é $x$.
	\end{itemize}
	onde $x\in \{p,c,g\}=\{\text{presidente}, \text{coordenadora}, \text{gerente}\}$. De acordo com o enunciado temos as seguintes premissas:

	Teste para os itens ($(a)-(b)$):
\begin{center}
		\[\begin{tabular}{lll}
			(1)\ &$C_g       \rightarrow B_p$ &\\
			(2)\ &$A_p       \rightarrow C_g$ &\\
			(3)\ &$\neg B_c  \rightarrow A_g$ &\\
			(4)\ &$C_c       \rightarrow B_g$ &\\
			(5)\ &$A_g      $\ &(\text{Teste a hipótese: Alice gerente}) \\
		         \hline
			(6)\ &$\neg A_p      $ & $5- P.N.C.$ \\
			(7)\ &$\neg A_c      $ & $5- P.N.C.$ \\
			(8)\ &$\neg B_g      $ & $5- P.N.C.$ \\
			(9)\ &$\neg C_g      $ & $5- P.N.C.$ \\
			(10)\ &$\neg C_c   $ & $4,8- MT$ \\
			(11)\ &$C_p   $ & $9,10- P.T.E.$ \\
			(12)\ &$B_c   $ & $5,11- P.T.E.$ \\
		\end{tabular}\]
		\end{center}
O argumento confirmar o item $(a)$ e torna falso o $(b)$.

	Teste para os itens ($(c)$):
\begin{center}
		\[\begin{tabular}{lll}
			(1)\ &$C_g       \rightarrow B_p$ &\\
			(2)\ &$A_p       \rightarrow C_g$ &\\
			(3)\ &$\neg B_c  \rightarrow A_g$ &\\
			(4)\ &$C_c       \rightarrow B_g$ &\\
			(5)\ &$A_c      $\ &(\text{Teste a hipótese: Alice coordenadora}) \\
		         \hline
			(6)\ &$\neg A_p      $ & $5- P.N.C.$ \\
			(7)\ &$\neg A_g      $ & $5- P.N.C.$ \\
			(8)\ &$\neg B_c      $ & $5- P.N.C.$ \\
			(9)\ &$\neg C_c      $ & $5- P.N.C.$ \\
			(10)\ &$\neg \neg B_c   $ & $3,7- MT$ \\
			(11)\ &$ B_c   $ & $10- DN$ \\
			(12)\ &$ \neg B_c\wedge B_c   $ & $8,11- CONJ$ \\
			\text{Contradição}
		\end{tabular}\]
		\end{center}
O sistema é inconsistente o que falso o $(c)$, pois Alice não pode ser coordenadora.
}

\newpage
\SOL{
	Teste para os itens ($(d)-(e)$):
\begin{center}
		\[\begin{tabular}{lll}
			(1)\ &$C_g       \rightarrow B_p$ &\\
			(2)\ &$A_p       \rightarrow C_g$ &\\
			(3)\ &$\neg B_c  \rightarrow A_g$ &\\
			(4)\ &$C_c       \rightarrow B_g$ &\\
			(5)\ &$A_p      $\ &(\text{Teste a hipótese: Alice presidente}) \\
		         \hline
			(6)\ &$\neg A_c      $ & $5- P.N.C.$ \\
			(7)\ &$\neg A_g      $ & $5- P.N.C.$ \\
			(8)\ &$\neg B_p      $ & $5- P.N.C.$ \\
			(9)\ &$\neg C_p      $ & $5- P.N.C.$ \\
			(10)\ &$\neg C_g   $ & $1,8- MT$ \\
			(10)\ &$\neg A_p   $ & $2,10- MT$ \\
			(11)\ &$ A_p\wedge \neg A_p   $ & $5,10- CONJ$ \\
			\text{Contradição}
		\end{tabular}\]
		\end{center}
O sistema é inconsistente o que torna falso o $(d)-(e)$, pois Alice não pode ser presidente.
}

\item
		 Considere a seguinte afirmação: Se Carlos é Eletricista, então Maria é Costureira e Marcelo é Escritor.
Assinale a alternativa que contém uma equivalência lógica para a afirmação apresentada.
\begin{enumerate}
		\item Se Maria não é Costureira e Marcelo não é Escritor, então Carlos não é Eletricista.
		\item Se Marcelo não é Escritor ou Maria não é Costureira, então Carlos não é Eletricista.
		\item Carlos é Eletricista e Maria é Costureira, e Marcelo é Escritor.
		\item Carlos é Eletricista, mas Marcelo não é Escritor ou Maria não é Costureira. 
		\item Carlos é Eletricista, mas Maria não é Costureira e Marcelo não é Escritor.
\end{enumerate}

\item 
		Se afino as cordas, então o instrumento soa bem. Se o instrumento soa bem, então toco muito bem. Ou não toco muito bem ou sonho acordado. Afirmo ser verdadeira a frase: não sonho acordado. Dessa forma, conclui-se que
		\begin{enumerate}
				\item  sonho dormindo.
				\item  o instrumento afinado não soa bem.
				\item  as cordas não foram afinadas.
				\item  mesmo afinado o instrumento não soa bem.
				\item  toco bem acordado e dormindo.
		\end{enumerate}

%Cespe-Sefaz-2024
\item
	No argumento seguinte, as proposições $P_1$, $P_2$, $P_3$ e $P_4$ são as premissas, e $Q$ é a
conclusão.
\begin{enumerate}
	\item $P_1$: ''Se há carência de recursos tecnológicos no setor Alfa, então o trabalho dos servidores públicos que atuam nesse setor ficam prejudicados''.
	\item $P_2$: ''Se há carência de recursos tecnológicos no setor Alfa, então os beneficiários dos serviços prestados por esse setor são mal atendidos''.
	\item $P_3$: ''Se o trabalho dos servidores públicos que atuam no setor Alfa fica prejudicado, então os servidores públicos que atuam nesse setor padecem''.
	\item $P_4$: “Se os beneficiários dos serviços prestados pelo setor Alfa são mal atendidos, então os beneficiários dos serviços prestados por esse setor padecem''.
	\item $Q$: ''Se há carência de recursos tecnológicos no setor Alfa, então os servidores públicos que atuam nesse setor padecem e os beneficiários dos serviços prestados por esse setor padecem''.
\end{enumerate}
Considerando esse argumento, julgue o item seguinte. O argumento em questão é válido? (Justifique)
%\begin{itemize}
%	\item $( )$ CERTO 
%	\item $( )$ ERRADO
%\end{itemize}

\item
	Ao comentar a respeito da qualidade dos
serviços de entrega de uma empresa, um aluno fez as seguintes afirmações:
\begin{itemize}
	\item $P_1$: Se for bom e rápido, não será barato.
	\item $P_2$: Se for bom e barato, não será rápido.
	\item $P_3$: Se for rápido e barato, não será bom.
\end{itemize}
Com base nessas informações, julgue.

Um argumento que tenha $P_1$ e $P_2$ como
premissas e $P_3$ como conclusão será um
argumento válido.

\item
	O famoso detetive Percule Hoirot foi chamado para resolver um assassinato misterioso. Ele determinou os
seguintes fatos:
\begin{enumerate}
	\item Lord Charles, o homem assassinado, foi morto com uma pancada na cabeça com um castiçal.
	\item  Ou Lady Camila ou a empregada Sara estavam na sala de jantar no momento do assassinato.
	\item Se o cozinheiro estava na cozinha no momento do assassinato, então o açougueiro matou Lord Charles com uma dose fatal de arsênico.
	\item Se Lady Camila estava na sala de jantar no momento do assassinato, então o motorista matou Lord Charles.
	\item Se o cozinheiro não estava na cozinha no momento do assassinato, então Sara não estava na sala de jantar quando o assassinato ocorreu.
		\item Se Sara estava na sala de jantar no momento do assassinato, então o ajudante pessoal de Lord Charles o matou.
\end{enumerate}
É possível para o detetive Percule Hoirot deduzir quem matou Lorde Charles? Se sim, quem é o assassino?

\SOL{Sejam os seguintes proposições:
	\begin{itemize}
		\item $p$ = Lord Charles foi morto com uma pancada na cabeça com um castiçal.
		\item $q$ = Lady Camila estava na sala de jantar no momento do assassinato.
		\item $r$ = Sara estava na sala de jantar no momento do assassinato.
		\item $s$ = Cozinheiro estava na cozinha no momento do assassinato.
		\item $t$ = Açougueiro matou Lord Charles com uma dose fatal de arsênico.
		\item $u$ = Motorista matou Lord Charles.
		\item $v$ = Ajudante pessoal de Lord Charles o matou.
	\end{itemize}
Os fatos podem ser rescritos simbolicamente:
\begin{center}
		\[\begin{tabular}{lll}
			(1)\ &$p$ &\\
			(2)\ &$q\vee r$ &\\
			(3)\ &$s \rightarrow t$ &\\
			(4)\ &$q \rightarrow u$ &\\
			(5)\ &$\neg s \rightarrow \neg r$ & \\
			(6)\ &$r \rightarrow v$\ & \\
			(7)\ &$?$\ & \text{Faça uma hipótese adicional!}\\
		         \hline
			(8)\ &$\neg t$\ & 1-P.N.C.\\
		\end{tabular}\]
		\end{center}


}

\item %(Rosen 1.5.3)
	Qual regra de inferência foi usada em cada um dos argumentos abaixo?
	\begin{enumerate}
		\item  ''Alice é uma aluna de matemática. Logo, Alice é uma aluna de matemática ou de ciência da computação.''
		\item ''Jerry é um aluno de matemática e de computação. Logo, Jerry é um aluno de matemática.''
		\item  ''Se está chovendo, então a piscina estará fechada. Está chovendo. Logo, a piscina está fechada.''
		\item ''Se nevar hoje, a universidade vai fechar. A universidade não fechou hoje. Logo, não nevou hoje.''
		\item ''Se eu for nadar, então eu ficarei no sol por muito tempo. Se eu ficar no sol por muito tempo, eu vou ter insolação. Logo, se eu for nadar, eu terei insolação. ''
	\end{enumerate}

}
%---------------------------------------------
%---------------------------------------------
\end{document}
